% --- GLOSSARY ENTRY --- 
% \newglossaryentry{latex}
% {
%     name=latex,
%     description={LaTeX is a document preparation system...}
% }
% --- ACRONYM ---
% \newacronym{it}{IT}{information technology}
% --- COMBINED GLOSSARY AND ACRONYM --- 
% \newglossaryentry{gui}
% {
%     name=GUI,
%     long={graphical user interface},
%     description={\Glsentrylong{gui}. ...},
%     first=\glsacronymnfirst{gui}
% }
% ---------------------------------------------------------------


\newglossaryentry{monolith}
{name=monolith, description={A software monolith or monolithic application
    describes an application that is built in a single unit, and that produces a
    single logical executable. This means any changes that are made to a part of
    the application require building and deploying a new version of the
    application \autocite{Fowler_Microservices_2014}}
}

\newglossaryentry{monolithic}
{
    name=monolithic,
    description={Built as a monolith. \textit{See} \gls{monolith}.}
}

\newglossaryentry{gui}
{
    name=GUI,
    long={graphical user interface},
    description={\Glsentrylong{gui}.},
    first=\glsacronymnfirst{gui}
}

\newglossaryentry{ui}
{
    name=UI,
    long={user interface},
    description={\Glsentrylong{ui}.},
    first=\glsacronymnfirst{ui}
}

\newglossaryentry{url}
{
    name=URL,
    long={Uniform Resource Locator},
    description={\Glsentrylong{url}. The address of a specific webpage or file on
    the internet.},
    first=\glsacronymnfirst{url}
}

\newglossaryentry{mvc}
{
    name=MVC, 
    long={model-view-controller}, 
    description={\Glsentrylong{mvc}. A design pattern, commonly used for user
    interfaces with the goal of seperating the user interface from the
    underlying data that represents it \autocite{Leff_Raylfield_2001}.}, 
    first=\glsacronymnfirst{mvc}
}

\newglossaryentry{frontend}
{
    name=frontend, 
    description={The frontend of a software application is the design,
    architecture and programming that makes the user-facing application
    function. } 
}

\newglossaryentry{backend}
{
    name=backend, 
    description={The backend of a software application is the part of a software
    system that is not directly accessed by the user, typically responsible for
    executing business logic and manipulating data.}
}

\newglossaryentry{api}
{
    name=API,
    long={application programming interface},
    description={An API or \glsentrylong{api} describes a set of commands and
    protocols for the communication between software systems without having to
    know their exact implementation. },
    first=\glsacronymnfirst{api}
}

\newglossaryentry{restful}
{
    name=RESTful,
    description={\textit{See} \gls{rest}.}
}

\newglossaryentry{rest}
{
    name=REST,
    long={representational state transfer},
    description={
    \Glsentrylong{rest} is an architectural style for distributed hypermedia
    systems. When a RESTful \glsentryname{api} is called, a transfer between the
    server and the client will occur that represents the state of the requested
    resource \autocite{Avraham_2017}.
    }
    first=REST % no full mention
}

\newglossaryentry{rpc}
{
    name=RPC,
    long={remote procedure call},
    description={
    A \Glsentrylong{rpc}, also known as a subroutine call or a function call, is
    a communication mechanism for client-server applications.
    },
    first=\glsacronymnfirst{rpc}
}

\newglossaryentry{endpoint}
{
    name=endpoint,
    description={
    Describes the point of entry in a communication channel. Relating to
    \glsplural{api}, an endpoint is the location or address where a request can
    be made to.
    }
}

\newglossaryentry{load-balancer}
{
    name=load-balancer, 
    description={
    A load enables the optimization of computing resources,
    reduces latency and increases output and the overall performance of a
    computing infrastructure, by distributing and managing the load across
    several devices \autocite{Techopedia_2012}.
    }
}

\newglossaryentry{ma}
{
    name={microservice architecture},
    description={A microservice architecture is a distributed application where
    all its modules are microservices. \autocite{Dragoni_etal_2017}.\textit{See
    also} \gls{microservice}.}
}

\newglossaryentry{microservice}
{
    name=microservice,
    description={A microservice is a cohesive, independent process interacting
    via messages \autocite{Dragoni_etal_2017}. Every microservice is loosely
    coupled, independently deployable and organized around a business
    capability. \textit{See also} \gls{ma}.} 
}

\newglossaryentry{soa}
{
    name=SOA, 
    long={service-oriented architecture},
    description={
    \Glsentrylong{soa} defines a way to make software components reusable and
    interoperable via service interfaces. Each service in an SOA embodies the
    code and data required to execute a business function \autocite{IBM_2021}.
    },
    first=\glsacronymnfirst{soa}
}

\newglossaryentry{http}
{
    name=HTTP,
    long={hypertext transfer protocol},
    description={The \Glsentrylong{http} is an application-level
    protocol for distributed, collaborative, hypermedia information
    systems. It is a generic, stateless, request/response protocol
    \autocite{Fielding_etal_1999}.
    },
    first=HTTP % no full mention
}

\newglossaryentry{soap}
{
    name=SOAP, 
    long={simple object access protocol},
    description={
    \Glsentrylong{soap}, a method of transferring messages formatted in XML over
    the Internet.
    },
    first=SOAP
}

\newglossaryentry{xml}
{
    name=XML,
    long={extensible markup language},
    description={\Glsentrylong{xml} is a markup language for representing
    structured information that is both human and machine-readable.},
    first=XML
}


\newglossaryentry{api_gateway}
{  
    name=API gateway, 
    description={
    An API gateway accepts API requests from a client, and directs them to the
    appropriate services. Typically it handles a request by invoking multiple
    microservices and aggregating the results \autocite{Nginx_2021}.
    }
}


\newglossaryentry{bff}
{
    name=BFF,
    long={Backend For Frontend},
    description={\Glsentrylong{bff}.},
    first=\glsacronymnfirst{bff}
}

\newglossaryentry{sso}
{
    name=SSO,
    long={single sign-on},
    description={\Glsentrylong{sso}.},
    first=\glsacronymnfirst{sso}
}

\newglossaryentry{mfa}
{
    name={microfrontend architecture}, 
    description={
    The microfrontend architecture is an architecture style that splits up an
    application into distributed modules that are focussed around a specific
    business capability. Most of the time these individual modules are managed
    by autonomous cross-functional teams \autocite{Geers_2020}\autocite{Rappl_2021}.
    }   
}

\newglossaryentry{microfrontend}
{
    name=microfrontend, 
    description={\textit{See} \gls{mfa}.}   
}

\newglossaryentry{gdd}
{
    name=GDD, 
    long={Geographically Distributed Development},
    description={
        \Glsentrylong{gdd}. The practice of managing software development projects
        beyond the traditional bounds of a single building or office structure where
        the development staff is singularly located. In a GDD model, the development
        staffing may be distributed across town, across a state or provincial
        border, or overseas \autocite{Yuhong_2008}.
    }, 
    first=\glsacronymnfirst{gdd}
}

\newglossaryentry{gdd2}
{
    name=GDD,
    long={Geographically Dispersed Development},
    description={\Glsentrylong{gdd2}. \textit{See} \glsentryname{gdd}.},
    first=\glsacronymnfirst{gdd2}
}

\newglossaryentry{gsd}
{
    name=GSD,
    long={Global Software Development},
    description={\Glsentrylong{gsd}. \textit{See} \glsentryname{gdd}.},
    first=\glsacronymnfirst{gsd}
}

\newglossaryentry{fts}
{
    name=FTS,
    long={follow-the-sun},
    description={\Glsentrylong{fts} (also called 24-hour development or
    round-the-clock development) is a form of \glsentryname{gdd} in which
    multiple teams are spread across timezones to ensure one team is always
    operational during normal business hours. },
    first=\glsacronymnfirst{fts}
}

\newglossaryentry{outsourcing}
{
    name=outsourcing, 
    description={The transfer of a business function to a third-party service provider.}
}

\newglossaryentry{offshoring}
{
    name=offshoring, 
    description={The tranfer of a business function to another country, usually
    to take advantage of lower labor costs, tax rates, or for legal reasons.}
}

\newglossaryentry{wasm}
{
    name=WebAssembly, 
    description={
        WebAssembly (abbreviated Wasm) is a binary instruction format for a
        stack-based virtual machine. Wasm is designed as a portable compilation
        target for programming languages, enabling deployment on the web for client
        and server applications \autocite{Webassembly_2021}.
    }
}

\newglossaryentry{w3c}
{
    name=W3C, 
    long={World Wide Web Consortium}, 
    description={The \Glsentrylong{w3c}
    is an international community where Member organizations, a full-time staff,
    and the public work together to develop Web standards \autocite{W3C_2021}.},
    first=\glsacronymnfirst{w3c}
}

\newglossaryentry{transpiling}
{
    name=transpiling, 
    description={
        Taking source code written in one language and transforming into another
    language that has a similar level of abstraction (high-level to
    high-level or low-level to low-level) \autocite{Fenton_2012}.
    }
}

\newglossaryentry{il}
{
    name=IL,
    long={intermediate language},
    description={
    \Glsentrylong{il}, also known as MSIL, is a product of compiling high-level
    .NET languages into a binary instruction format \autocite{Microsoft_2016}.},
    first=\glsacronymnfirst{il}
}

\newglossaryentry{dom}
{
    name=DOM, 
    long={Document Object Model}, 
    description={\Glsentrylong{dom}, a programming interface for web documents.
    The DOM represents the document as nodes and objects; that way, programming
    languages can interact with the page \autocite{Mozilla_DOM}.}, 
    first=\glsacronymnfirst{dom}}

\newglossaryentry{jsinterop}
{
    name={JS interop},
    long={Javascript interoperability},
    description={\Glsentrylong{jsinterop} describes the invocation of a
    JavaScript code from another language.},
    first=\glsacronymnfirst{jsinterop}
}

\newglossaryentry{pe}
{
    name={progressive enhancement},
    description={
    A software design philosophy that emphasises the delivery of simple content
    and basic funtionality to as many users as possible, while providing the
    best user experience only to the users of capable systems and browsers.
    }
}

\newglossaryentry{seo}
{
    name=SEO,
    long={search engine optimization},
    description={\Glsentrylong{seo}.},
    first=\glsacronymnfirst{seo}
}

\newglossaryentry{iframe}
{
    name=iframe,
    description={Inline frame.}
}

\newglossaryentry{poc}
{
    name=PoC,
    long={proof of concept},
    description={\Glsentrylong{poc}.},
    first=\glsacronymnfirst{poc}
}

\newglossaryentry{appshell}
{
    name={application shell},
    description={
    Also called \texit{app shell}. It serves as a parent application for the
    integration of microfrontends \autocite{Geers_2020}\autocite{Rappl_2021}.
    }
}

\newglossaryentry{cli}
{
    name=CLI,
    long={Command Line Interface},
    description={\Glsentrylong{cli}.},
    first=\glsacronymnfirst{cli}
}

\newglossaryentry{di}
{
    name=DI,
    long={dependency injection},
    description={\Glsentrylong{di}.},
    first=\glsacronymnfirst{di}
}

\newglossaryentry{cdn}
{
    name=CDN,
    long={Content Delivery Network},
    description={
    A \Glsentrylong{cdn}, is a distributed network of servers that can
    efficiently deliver web content to users \autocite{Microsoft_2018}.
        },
    first=\glsacronymnfirst{cdn}
}

\newglossaryentry{dll}
{
    name=DLL,
    long={dynamic-link library},
    description={\Glsentrylong{dll}.},
    first=\glsacronymnfirst{dll}
}

\newglossaryentry{pdb}
{
    name=PDB,
    long={program database},
    description={\Glsentrylong{pdb}.},
    first=\glsacronymnfirst{pdb}
}

\newglossaryentry{ssi}
{
    name=SSI,
    long={Server-Side Includes},
    description={\Glsentrylong{ssi}.},
    first=\glsacronymnfirst{ssi}
}

\newglossaryentry{esi}
{
    name=ESI,
    long={Edge-Side Includes},
    description={\Glsentrylong{esi}.},
    first=\glsacronymnfirst{esi}
}

\newglossaryentry{spa}
{
    name=SPA,
    long={Single-page application},
    description={\Glsentrylong{spa}.},
    first=\glsacronymnfirst{spa}
}

\newglossaryentry{ux}
{
    name=UX,
    long={user experience},
    description={\Glsentrylong{ux}.},
    first=\glsacronymnfirst{ux}
}
