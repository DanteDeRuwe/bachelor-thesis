\section{Distributed development}

In a technology landscape that has long been shifting to meet the current trends
of economic globalization, software development has evolved from being mostly
concentrated at a single location, to being geographically distributed around
the globe. 

According to \textcite{Yuhong_2008}, the terms \gls{gdd}, \gls{gdd2} and
\gls{gsd} are mostly used to refer to the same distributed development model. In
the remainder of this thesis, the term \glsacronymnfirst{gdd} will be used mainly
when emphasis is required on the geographical aspect of the distributed
development model.

The application of \gls{gdd} can be done in many distinct ways. One way is in
the form of \textbf{\gls{outsourcing}} agreements, often with countries where
employment costs are more economically beneficial.

Another way, is the separation of a company into different local
\textbf{divisions} or departments in different cities. This envelops both large
multinational companies that operate around the globe, or more
nationally-focused companies that have different branches in different cities.
\autocite{Kiel_2003} This also includes the practice of \textbf{\gls{offshoring}},
which tends to have the same motivations as outsourcing, but keeps control in
the hands of the business, and does not involve a third party \autocite{Oshri_2015}.


\subsection{Benefits}

Companies often have a plethora of different business reasons to distribute the
development, maintenance and management of software. 

\begin{itemize}
    \item Financial benefit (labor costs, taxation, ...)
    \item Market insight: local teams have more insight into local trends
    \item Talent availability: a larger pool of skilled developers is available,
    potentially even with different specializations in different locations.
    \autocite{Conchuir_etal_2009}
    \item Faster time-to-market: because development can be distributed across
    multiple timezones, a \gls{fts} workflow could potentially increase
    development speed drastically\footnote{According to
    \textcite{Conchuir_etal_2009}, \gls{fts} workflow is however practically
    almost inachievable, and in practice, many companies even make sure the time
    zones overlap as much as possible, to reach better inter-team communication.
    } \autocite{Carmel_2010}.
\end{itemize}

\subsection{Challenges}

While the aforementioned benefits of \gls{gdd} appear very useful in theory, on
the practical side, there are of course some challenges to this approach. For
example, according to \textcite{Smite_etal_2010}, implementing an \textbf{agile
methodology} within a distributed software development model is not
straightforward; and the characteristics of agile and distributed development
could be seen as polar opposites. 

But the idiomatic \textit{``elephant in the room''} is \textbf{communication}.
In projects where teams are located together, communication can be rather
informal, which helps team members more rapidly gain project and technical
knowledge, as well as knowledge of the more human aspects of their coworkers,
such as working style and expertise. According to \textcite{Sengupta_2006},
frequency of communication has an inverse relationship to physical separation of
team-members, and in multi-site environments, the decrease in communication
frequency is so sharp, that informal communication is nearly nonexistent.
Pairing this with \textbf{cultural} and timezone differences, makes all
communication very difficult when practicing \gls{gdd}.


\subsection{The role of the architecture in enabling distributed development}

According to \textcite{Yuhong_2008}, in a \gls{gdd} environment, establishing
and maintaining a common software and/or solution architecture that can support
a distributed development model, is key for the success and sustainability of
the software project. Among other architectures, she describes a module-based
project architecture, where self-contained software components are developed
independently. This way, teams could develop these modules simultaneously,
without a large interdependence on other modules. This, however, requires strong
decoupling of the software modules.