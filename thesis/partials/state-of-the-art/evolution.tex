\section{The evolution from monolithic to distributed architectures}

The development of applications for the web has seen some dramatic shifts over
the years. Apart from new technologies, protocols, and standards, the way web
applications are structured has undergone some evolutions as well. ``Software
architecture'' not only outlines the pure structure of the application but also
defines the responsibility of all the pieces of the application, and how these
pieces ought to interact with each other \autocite{Fedorov_etal_1998}.


\subsection{client-server model}

Historically, one of the earliest architectures for implementing web
applications was the \textbf{client-server model}. The service provider or
\textit{server} can share its resources with service users called
\textit{clients}. According to \textcite{Reese_2000}, at the time, most web
applications were simple two-tier client-server applications. The web browser on
the client-side retrieves data and files from the data store at the webserver
side, without much data interpretation or manipulation. The upcoming increase of
the computing power of hardware, made it possible to execute some data
processing on the client-side, using for example technologies like Java. 

Although two-tier server-client architectures were quick to set up and had
robust tooling, a pretty significant downside got introduced: these so-called
\textit{fat clients} were now not only concerned with the task of presenting
data to the user, but are also bloated with business logic and data processing.
Conversely, any change in business rules would also require every client to adapt
to this change. \autocite{Gallaugher_Ramanathan_1996}.

A three-tier architecture was conceptualized in an attempt to overcome the
downsides of the two-tier approach. The idea sounds not very groundbreaking:
just introduce the handling of business logic on the server-side, rather than on
the client. This results in the following tiers \autocite{Aarsten_etal_1996}:

\begin{itemize}
    \item The client tier (also known as the presentation tier), which contains
    the graphical user interface (GUI)
    \item The application tier (also known as the business tier or logic tier),
    i.e. the application servers that contain objects representing business
    entities and domain logic.
    \item The data(base) tier that handles the storage of domain objects and data.
\end{itemize}



% two-tier vs three tier figure here? 





