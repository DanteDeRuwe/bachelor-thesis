\section{Microfrontends}

With the introduction of the \gls{ma} pattern, \gls{backend} systems can be
split up into multiple services, each with their own responsibilities. As
previously described, this can bring great benefit. However, even after a
transition to \glsplural{microservice} on the server-side, the client-side
applications using these services are mostly still \gls{monolithic} in nature.

A \gls{monolithic} \gls{frontend} does not have to be a problem.
\Glsplural{monolith} are quick and easy to set up, and historically, most of the
heavy lifting was done on the server-side anyway. However, the complexity of
client-side applications has seen a drastic increase over the last few years.
This can be attributed to many factors: increased hardware and web browser
capabilities, a broad variety of client devices, a massive market growth for
digital services, and the web transitioning from a document platform to the
largest application platform \autocite{Ball_2019}, just to name a few.

In these complex applications, the downsides of a \gls{monolithic} architecture
come back into view: every change requires the entire \gls{frontend} application
to be rebuilt and redeployed, codebases grow very large in size, etc... Even
worse, because the client-side application has a functional dependency on the
server-side application, a small change in one particular area of the
\gls{backend} logic could also trigger a change in the \gls{frontend}, causing
the entire \gls{frontend} to yet again be rebuilt and redeployed
\autocite{Rappl_LogRocket_2019}.

There's also the issue of \textbf{domain knowledge}: while the \gls{ma} gives
\gls{backend} teams the possibility of focussing on one specific part of a
business domain, teams that are developing the client-side code are still
expected to know the entire scope of the application. Often this means a
reliance on personal inter-team communication, which in larger organizations
tends to be expensive \autocite{Geers_2020}. 

While component-based paradigms introduced by libraries and frameworks (such as
React\hreffootnote{https://reactjs.org},
Angular\hreffootnote{https://angular.io} and
Vue\hreffootnote{https://vuejs.org}) can aliviate some of the complexity of the
current frontend systems, they still do not enable  fully autonomous, decoupled,
modular and/or distributed development of large web applications.

\subsection{What are microfrontends?}

In 2016 the ThoughtWorks Technology Radar \autocite{ThoughtWorks_2020} coined
the term ``\textit{Micro Frontends}'' to describe the split of the
\gls{frontend} \gls{monolith} into independently deployable and maintainable
pieces. This new architecture pattern could therefore be regarded as an
extension of the \gls{ma} into the \gls{frontend} space.

The characteristics of the \gls{mfa} pattern are therefore very closely related
to those of the \gls{ma} pattern, as described in section
\fullref{principles-of-the-mfa}. Every individual \gls{frontend} module has a
relatively small codebase, is focussed around one specific domain or company
mission, and should be modular, decoupled and independently developed; most
optimally by autonomous teams.

\subsection{How do microfrontends enable distributed development?}

\subsubsection{feature teams}
\label{feature-teams}
...

\subsection{Common implementation patterns}

As is the case with lots of architectural patterns, there are many ways of
implementation possible. With the \gls{mfa}, this is no different. Various
options differ in complexity, goal and mechanism. Below some of the most common
ways of implementation are outlined.

\subsubsection{The \textit{web approach}}
...
\subsubsection{Server-side rendering}
...
\subsubsection{Client-side rendering}
...
\subsubsection{Universal rendering}
...

\subsection{Usage of microfrontends}
...

\subsubsection{benefits}

The \gls{mfa} carries over a lot of the advantages of the \gls{ma}
\autocite{Jackson_2019}: \textbf{Technology independence} is now also possible
across different \gls{frontend} teams, allowing them to select the best tools
and frameworks for the job. \textbf{Teams} also benefit greatly, and can be
reorganized to even greater benefit, as was discussed in section
\fullref{feature-teams}.

...
% TODO more

\subsubsection{known limitations and caveats}


However, to use the words of Cam \textcite{Jackson_2019}: \textit{``there are no
free lunches when it comes to software architecture - everything comes with a
cost''}. The \gls{mfa} does indeed come with a significant amount of tradeoffs
that need to be considered. 

...
% TODO more