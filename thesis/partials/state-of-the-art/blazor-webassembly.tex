\section{Blazor WebAssembly}

On the 6th of February 2018, Daniel Roth -- Program Manager on the ASP.NET team
at Microsoft -- released a blog post called \textit{A new experiment:
Browser-based web apps with .NET and Blazor}. In this post, \textcite{Roth_2018}
announces an experimental project from the ASP.NET team: a component-ori\"ented
web \gls{ui} framework based on C\#, .NET, HTML and so-called Razor pages. 

The promise that was outlined by this post was a way to enable developers to
write web applications using .NET technologies, rather than resorting to
Javascript\footnote{Javascript is an implementation of the ECMAScript
specification. Read more on \hrefself{https://ecma-international.org/tc39}}, the
primary scripting language used on the web.

Executing .NET binaries within a web browser is made possible by
\textbf{\gls{wasm}}\hreffootnote{https://webassembly.org/}, a binary instruction format.
\Gls{wasm} has been added to the \gls{w3c} recommendation list, and has become
the fourth language to run natively in web browsers, alongside HTML, CSS and
Javascript \autocite{Couriol_2019}. 

Rather than \gls{transpiling} every .NET assembly to \gls{wasm}, or relying on
plugins, Blazor just relies on this .NET runtime that can run inside the browser
sandbox, just like regular Javascript does. Because of this applications can
leverage all standard web technologies like websockets, the \gls{dom}, and all
other browser \glsplural{api}, via \textbf{\gls{jsinterop}}. It also ensures the
various security protections put in place by the sandbox environment to prevent
malicious client-side attacks.

The current implementation uses the \gls{wasm}-compiled version of the
Mono\hreffootnote{https://mono-project.com/} platform -- an open source .NET
runtime -- as an \gls{il} interpreter to execute managed code at runtime.


% TODO diagram of blazor, mono, js runtime, ......
\dots %TODO


\subsection{Current state of Blazor and microfrontends}
\dots %TODO

\subsection{Challenges and limitations}
\dots %TODO