\section{Blazor WebAssembly}

On the 6th of February 2018, Daniel Roth -- Program Manager on the ASP.NET team
at Microsoft -- released a blog post called \textit{A new experiment:
Browser-based web apps with .NET and Blazor}. In this post, \textcite{Roth_2018}
announces an experimental project from the ASP.NET team: a component-ori\"ented
web \gls{ui} framework based on C\#, .NET, HTML and so-called Razor pages. 

The promise that was outlined by this post was a way to enable developers to
write web applications using .NET technologies, rather than resorting to
Javascript\footnote{Javascript is an implementation of the ECMAScript
specification. Read more on \hrefself{https://ecma-international.org/tc39}}, the
primary scripting language used on the web.

Executing .NET binaries within a web browser is made possible by
\textbf{\gls{wasm}}\hreffootnote{https://webassembly.org/}, a binary instruction format.
\Gls{wasm} has been added to the \gls{w3c} recommendation list, and has become
the fourth language to run natively in web browsers, alongside HTML, CSS and
Javascript \autocite{Couriol_2019}. 

An overview of how Blazor works is provided in Figure~\ref{fig:blazor-wasm}.

Rather than \gls{transpiling} every .NET assembly to \gls{wasm}, or relying on
plugins, Blazor just relies on a .NET runtime that can run inside the browser
sandbox, just like regular Javascript does. The current implementation of Blazor
uses the \gls{wasm}-compiled version of the
Mono\hreffootnote{https://mono-project.com/} platform -- an open source .NET
runtime -- as an \gls{il} interpreter to execute managed code at runtime.

Because of this, applications can leverage all standard web technologies like
websockets, the \gls{dom}, and all other browser \glsplural{api}, via
\textbf{\gls{jsinterop}}. It also ensures the various security protections put
in place by the sandbox environment to prevent malicious client-side attacks.

\begin{figure}
  \centering
  \includegraphics[width=0.8\textwidth]{blazor-wasm}
  \caption[Blazor WebAssembly]{A diagram that outlines how Blazor \gls{wasm}
  works. The C\# code and Razor files get compiled into \gls{il} assemblies, and
  run on a \gls{wasm} version of the Mono platform -- a .NET runtime -- inside
  the browser sandbox.}
  \label{fig:blazor-wasm}
\end{figure}


\subsection{Current state of Blazor and microfrontends}

According to \textcite{Rappl_MunichNETMeetup_2020}, an easy and effective way to
make distributed development possible in a Blazor \gls{wasm} project, is simply
to use separately distributed \textbf{component libraries}. In Blazor -- and in
the .NET ecosystem in general -- NuGet\hreffootnote{https://www.nuget.org} is
the standardized way of library distribution. One could create an \gls{appshell}
which imports and incorporates these NuGet packages. While this implementation
pattern makes development relatively straightforward, it has its downsides.

Because the integration happens at build-time; any change requires full
recompilation of the entire application. Also, upon startup, the \gls{appshell}
has to have full knowledge of all libraries, and they all have to be loaded for
the application to work, which makes this approach suboptimal regarding
scalability. This integration method re-introduces coupling, and is generally
discouraged \autocite{Jackson_2019}.

Integrating \textbf{Blazor components in a JavaScript-based app shell} is
another option. While this has been done succesfully\footnote{See an example
here: \hrefself{https://github.com/lauchacarro/MicroFrontend-Blazor-React}}, and
there are already some frameworks that support this idea\footnote{For example
the Piral \gls{microfrontend} framework with their \texttt{piral-blazor} converter.
See \url{https://piral.io}}, it is not within the scope of this thesis, as the
focus is on an almost exclusive .NET-approach.

\subsubsection{Challenges and potential solutions}

When looking for an approach that can dynamically load assemblies at runtime,
the client-side \textbf{routing} starts to present a problem. When defining a
page component in Blazor, the \texttt{@page} directive can be used. This will
later provide the generated class with a \texttt{RouteAttribute} with a value
that indicates the component's desired route template. The standard Blazor
router will then use reflection on the specified \texttt{AppAssembly} to scan the
loaded assemblies for these \texttt{RouteAttribute}s \autocite{Sainty_2019}.
This starts to become an issue if the assemblies are dynamically loaded -- and
thus not present upon compilation. 

Another challenge that can present itself is \textbf{debugging}. When running
Blazor DLLs on a \gls{wasm} runtime, a mechanism is needed to provide the link
between the browser and the debugging tools: the debugging proxy. According to
\textcite{Abdalla_2020}, this is a separate process that gets launched to load
so-called \gls{pdb} files. These are also called \textit{symbol} files, and they
are the link between the debugger and the source code \autocite{Microsoft_2021}.
Loading the \gls{microfrontend} symbol files dynamically is a challenge that
would need to be overcome to allow the debugging of Blazor applications using
\glsplural{microfrontend}.

These are challenges that would not come up when the composition would be done
at build-time like with component libraries. Luckily, the .NET 5 version of
Blazor introduces lazy-loading capabilities out of the
box\hreffootnote{https://docs.microsoft.com/en-us/aspnet/core/blazor/webassembly-lazy-
load-assemblies}, so most of these challenges that come from the dynamic loading
requirement can most likely be overcome. According to \textcite{Kdouh_2020}, the
Blazor router component, for example, now supports passing it additional
assemblies to consider. For debugging, the \texttt{AssemblyLoadContext} can
support the dynamic loading of dependencies with their symbols
\autocite{Microsoft_2019}. 