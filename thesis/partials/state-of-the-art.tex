% ----------------------------------------
% State of the art
% ----------------------------------------

\chapter{State of the art}
\label{ch:state-of-the-art}

% Begin elk hoofdstuk met een paragraaf inleiding die beschrijft hoe dit
% hoofdstuk past binnen het geheel van de bachelorproef. Geef in het bijzonder
% aan wat de link is met het vorige en volgende hoofdstuk.

% Pas na deze inleidende paragraaf komt de eerste sectiehoofding.

% Dit hoofdstuk bevat je literatuurstudie. De inhoud gaat verder op de inleiding,
% maar zal het onderwerp van de bachelorproef *diepgaand* uitspitten. De bedoeling
% is dat de lezer na lezing van dit hoofdstuk helemaal op de hoogte is van de
% huidige stand van zaken (state-of-the-art) in het onderzoeksdomein. Iemand die
% niet vertrouwd is met het onderwerp, weet nu voldoende om de rest van het
% verhaal te kunnen volgen, zonder dat die er nog andere informatie moet over
% opzoeken.


In this chapter, the research domain will be explored in its current state. This
will allow for further insight into what distributed development is and why it
is beneficial. The evolution of software architectures from monolithic to
distributed will also be laid out, which introduces the concept of
the \gls{mfa}. To conclude, a deep-dive into Blazor \Gls{wasm} will be presented.

\section{Distributed development}

In a technology landscape that has long been shifting to meet the current trends
of economic globalization, software development has evolved from being mostly
concentrated at a single location, to being geographically distributed around
the globe. 

According to \textcite{Yuhong_2008}, the terms \gls{gdd}, \gls{gdd2} and
\gls{gsd} are mostly used to refer to the same distributed development model. In
the remainder of this thesis, the term \glsacronymnfirst{gdd} will be used mainly
when emphasis is required on the geographical aspect of the distributed
development model.

The application of \gls{gdd} can be done in many distinct ways. One way is in
the form of \textbf{\gls{outsourcing}} agreements, often with countries where
employment costs are more economically beneficial.

Another way, is the separation of a company into different local
\textbf{divisions} or departments in different cities. This envelops both large
multinational companies that operate around the globe, or more
nationally-focused companies that have different branches in different cities.
\autocite{Kiel_2003} This also includes the practice of \textbf{\gls{offshoring}},
which tends to have the same motivations as outsourcing, but keeps control in
the hands of the business, and does not involve a third party \autocite{Oshri_2015}.


\subsection{Benefits}

Companies often have a plethora of different business reasons to distribute the
development, maintenance and management of software. 

\begin{itemize}
    \item Financial benefit (labor costs, taxation, ...)
    \item Market insight: local teams have more insight into local trends
    \item Talent availability: a larger pool of skilled developers is available,
    potentially even with different specializations in different locations.
    \autocite{Conchuir_etal_2009}
    \item Faster time-to-market: because development can be distributed across
    multiple timezones, a \gls{fts} workflow could potentially increase
    development speed drastically\footnote{According to
    \textcite{Conchuir_etal_2009}, \gls{fts} workflow is however practically
    almost inachievable, and in practice, many companies even make sure the time
    zones overlap as much as possible, to reach better inter-team communication.
    } \autocite{Carmel_2010}.
\end{itemize}

\subsection{Challenges}

While the aforementioned benefits of \gls{gdd} appear very useful in theory, on
the practical side, there are of course some challenges to this approach. For
example, according to \textcite{Smite_etal_2010}, implementing an \textbf{agile
methodology} within a distributed software development model is not
straightforward; and the characteristics of agile and distributed development
could be seen as polar opposites. 

But the ideomatic \textit{``elephant in the room''} is \textbf{communication}.
In projects where teams are located together, communication can be rather
informal, which helps team members more rapidly gain project and technical
knowledge, as well as knowledge of the more human aspects of their coworkers,
such as working style and expertise. According to \textcite{Sengupta_2006},
frequency of communication has an inverse relationship to physical seperation of
team-memebers, and in multi-site environments, the decrease in communication
frequency is so sharp, that informal communication is nearly nonexistent.
Pairing this with \textbf{cultural} and timezone differences, makes all
communication very difficult when practicing \gls{gdd}.


\subsection{The role of the architecture in enabling distributed development}

According to \textcite{Yuhong_2008}, in a \gls{gdd} environment, establishing
and maintaining a common software and/or solution architecture that can support
a distributed development model, is key for the success and sustainability of
the software project. Among other architectures, she describes a module-based
project architecture, where self-contained software components are developed
independently. This way, teams could develop these modules simultaneously,
without a large interdependence on other modules. This however, requires strong
decoupling of the software modules.
\section{The evolution from monolithic to distributed architectures}

The development of applications for the web has seen some dramatic shifts over
the years. Apart from new technologies, protocols and standards, the way web
applications are structured has undergone some evolutions as well. ``Software
architecture'' not only outlines the pure structure of the application, but also
defines the responsibility of all the pieces of the application, and how these
pieces ought to interact with each other \autocite{Fedorov_etal_1998}.


\subsection{client-server model}

Historically, one of the earliest architectures for implementing web
applications was the \textbf{client-server model}. The service provider or
\textit{server} can share its resources with service users called
\textit{clients}. According to \textcite{Reese_2000}, at the time, most web
applications were simple two-tier client-server applications. The web browser on
the client-side retrieves data and files from the data store at the web server
side, without much data interpretation or manipulation. The upcoming increase of
the computing power of hardware, made it possible to execute some data
processing on the client-side, using for example technologies like Java. 

Although two-tier server-client architectures were quick to set up and had
robust tooling, a pretty significant downside got introduced: these so-called
\textit{fat clients} were now not only concerned with the task of presenting
data to the user, but are also bloated with business logic and data processing.
Coversely, any change in business rules would also require every client to adapt
to this change. \autocite{Gallaugher_Ramanathan_1996}.

A three-tier architecture was conceptualized in an attempt to overcome the
downsides of the two-tier approach. The idea sounds not very groundbreaking:
just introduce the handling of business logic on the server-side, rather than on
the client. This results in the following tiers \autocite{Aarsten_etal_1996}:

\begin{itemize}
    \item The client tier (also known as the presentation tier), which contains
    the graphical user interface (GUI)
    \item The application tier (also known as the business tier or logic tier),
    i.e the application servers that contain objects representing business
    enties and domain logic.
    \item The data(base) tier that handles the storage of domain objects and data.
\end{itemize}



% two-tier vs three tier figure here? 






\section{Microfrontends}

With the introduction of the \gls{ma} pattern, \gls{backend} systems can be
split up into multiple services, each with their own responsibilities. As
previously described, this can bring great benefit. However, even after a
transition to \glsplural{microservice} on the server-side, the client-side
applications using these services are mostly still \gls{monolithic} in nature.

A \gls{monolithic} \gls{frontend} does not have to be a problem.
\Glsplural{monolith} are quick and easy to set up, and historically, most of the
heavy lifting was done on the server-side anyway. However, the complexity of
client-side applications has seen a drastic increase over the last few years.
This can be attributed to many factors: increased hardware and web browser
capabilities, a broad variety of client devices, a massive market growth for
digital services, and the web transitioning from a document platform to the
largest application platform \autocite{Ball_2019}, just to name a few.

In these complex applications, the downsides of a \gls{monolithic} architecture
come back into view: every change requires the entire \gls{frontend} application
to be rebuilt and redeployed, codebases grow very large in size, etc... Even
worse, because the client-side application has a functional dependency on the
server-side application, a small change in one particular area of the
\gls{backend} logic could also trigger a change in the \gls{frontend}, causing
the entire \gls{frontend} to yet again be rebuilt and redeployed
\autocite{Rappl_LogRocket_2019}.

There's also the issue of \textbf{domain knowledge}: while the \gls{ma} gives
\gls{backend} teams the possibility of focussing on one specific part of a
business domain, teams that are developing the client-side code are still
expected to know the entire scope of the application. Often this means a
reliance on personal inter-team communication, which in larger organizations
tends to be expensive \autocite{Geers_2020}. 

While component-based paradigms introduced by libraries and frameworks (such as
React\hreffootnote{https://reactjs.org},
Angular\hreffootnote{https://angular.io} and
Vue\hreffootnote{https://vuejs.org}) can aliviate some of the complexity of the
current frontend systems, they still do not enable  fully autonomous, decoupled,
modular and/or distributed development of large web applications.

\subsection{What are microfrontends?}

In 2016 the ThoughtWorks Technology Radar \autocite{ThoughtWorks_2020} coined
the term ``\textit{Micro Frontends}'' to describe the split of the
\gls{frontend} \gls{monolith} into independently deployable and maintainable
pieces. This new architecture pattern could therefore be regarded as an
extension of the \gls{ma} into the \gls{frontend} space.

The characteristics of the \gls{mfa} pattern are therefore very closely related
to those of the \gls{ma} pattern, as described in section
\fullref{principles-of-the-mfa}. Every individual \gls{frontend} module has a
relatively small codebase, is focussed around one specific domain or company
mission, and should be modular, decoupled and independently developed; most
optimally by autonomous teams.

\subsection{How do microfrontends enable distributed development?}

\subsubsection{feature teams}
\label{feature-teams}
...

\subsection{Common implementation patterns}

As is the case with lots of architectural patterns, there are many ways of
implementation possible. With the \gls{mfa}, this is no different. Various
options differ in complexity, goal and mechanism. Below some of the most common
ways of implementation are outlined.

\subsubsection{The \textit{web approach}}
...
\subsubsection{Server-side rendering}
...
\subsubsection{Client-side rendering}
...
\subsubsection{Universal rendering}
...

\subsection{Usage of microfrontends}
...

\subsubsection{benefits}

The \gls{mfa} carries over a lot of the advantages of the \gls{ma}
\autocite{Jackson_2019}: \textbf{Technology independence} is now also possible
across different \gls{frontend} teams, allowing them to select the best tools
and frameworks for the job. \textbf{Teams} also benefit greatly, and can be
reorganized to even greater benefit, as was discussed in section
\fullref{feature-teams}.

...
% TODO more

\subsubsection{known limitations and caveats}


However, to use the words of Cam \textcite{Jackson_2019}: \textit{``there are no
free lunches when it comes to software architecture - everything comes with a
cost''}. The \gls{mfa} does indeed come with a significant amount of tradeoffs
that need to be considered. 

...
% TODO more
\section{Blazor WebAssembly}

On the 6th of February 2018, Daniel Roth -- Program Manager on the ASP.NET team
at Microsoft -- released a blog post called \textit{A new experiment:
Browser-based web apps with .NET and Blazor}. In this post, \textcite{Roth_2018}
announces an experimental project from the ASP.NET team: a component-ori\"ented
web \gls{ui} framework based on C\#, .NET, HTML and so-called Razor pages. 

The promise that was outlined by this post was a way to enable developers to
write web applications using .NET technologies, rather than resorting to
Javascript\footnote{Javascript is an implementation of the ECMAScript
specification. Read more on \hrefself{https://ecma-international.org/tc39}}, the
primary scripting language used on the web.

Executing .NET binaries within a web browser is made possible by
\textbf{\gls{wasm}}\hreffootnote{https://webassembly.org/}, a binary instruction format.
\Gls{wasm} has been added to the \gls{w3c} recommendation list, and has become
the fourth language to run natively in web browsers, alongside HTML, CSS and
Javascript \autocite{Couriol_2019}. 

Rather than \gls{transpiling} every .NET assembly to \gls{wasm}, or relying on
plugins, Blazor just relies on this .NET runtime that can run inside the browser
sandbox, just like regular Javascript does. Because of this applications can
leverage all standard web technologies like websockets, the \gls{dom}, and all
other browser \glsplural{api}, via \textbf{\gls{jsinterop}}. It also ensures the
various security protections put in place by the sandbox environment to prevent
malicious client-side attacks.

The current implementation uses the \gls{wasm}-compiled version of the
Mono\hreffootnote{https://mono-project.com/} platform -- an open source .NET
runtime -- as an \gls{il} interpreter to execute managed code at runtime.


% TODO diagram of blazor, mono, js runtime, ......

...


\subsection{Current state of Blazor and microfrontends}
...

\subsection{challenges and limitations}
...
