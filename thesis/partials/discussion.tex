% ----------------------------------------
% Discussion
% ----------------------------------------

\chapter{Discussion}
\label{ch:discussion}

% Trek een duidelijke conclusie, in de vorm van een antwoord op de
% onderzoeksvra(a)g(en). Wat was jouw bijdrage aan het onderzoeksdomein en
% hoe biedt dit meerwaarde aan het vakgebied/doelgroep? 
% Reflecteer kritisch over het resultaat. In Engelse teksten wordt deze sectie
% ``Discussion'' genoemd. Had je deze uitkomst verwacht? Zijn er zaken die nog
% niet duidelijk zijn?
% Heeft het onderzoek geleid tot nieuwe vragen die uitnodigen tot verder 
%onderzoek?

The \gls{mfa} can provide development teams with significant benefits, if they
are willing to accept the drawbacks and combat the challenges that go along with
its implementation. 

The modular character of this architecture pattern can in theory be one of the
most effective ways to achieve loosely coupled end-to-end services
(\textit{``from database to \gls{ui}'')}, which can be developed and managed by
autonomous cross-functional teams. This can enable development teams to be
(geographically) distributed. Another benefit of operating autonomously, is the
reduced waiting time and faster release cycles for the individual modules.

However, this organizational shift might in some cases not be easy or even
possible. Especially in smaller companies, the advantages and disadvantages
should carefully be weighed.

On the technical side, creating a solid \gls{microfrontend} solution has its
challenges too. Creating a robust application all starts with a good conceptual
and architectural baseline. This is no different when applying the \gls{mfa} to
the Blazor \gls{wasm} technology. To be able to achieve very loose coupling of
the different modules, a dynamic loading strategy of the needed assemblies can
be necessary. This introduces the challenges of routing and fragment rendering,
both of which were addressed by the proof-of-concept solution found in
Chapter~\fullref{ch:proof-of-concept}.

Creating a real-world \gls{microfrontend} solution within the .NET ecosystem
using Blazor \gls{wasm} is possible, but like everything, can be improved.
Further research and experimental work can be done on providing a way to enable
every microfrontend to bring its own dependencies and static files, while still
ensuring isolation and performance, and avoiding runtime conflicts. Also
research in how to develop the microfrontends seperately from the full
\gls{appshell} can be helpful. One way of potentially achieving this is to
bundle up the \gls{appshell} into some kind of emulator to enable local
development. This would also enable the local debugging of these
\glsplural{microfrontend} without the need for the full \gls{appshell}.

Without much doubt, the future versions of Blazor will bring more possibilities
to solve the problems and challenges that come up when developing
\glsplural{microfrontend} with Blazor. Performance keeps improving, and new ways
of dynamic loading and server-side rendering are being introduced. 

While some form of \glsplural{microfrontend} have made their introduction in
large companies recently, the architecture pattern is just slowly breaking into
the \textit{mainstream} development community. It will be interesting to see
what solutions and frameworks will come up to make development of these modular
applications easier than ever before.
