% ----------------------------------------
% Introduction
% ----------------------------------------

\chapter{Introduction}
\label{ch:introduction}

% De inleiding moet de lezer net genoeg informatie verschaffen om het onderwerp
% te begrijpen en in te zien waarom de onderzoeksvraag de moeite waard is om te
% onderzoeken. In de inleiding ga je literatuurverwijzingen beperken, zodat de
% tekst vlot leesbaar blijft. Je kan de inleiding verder onderverdelen in
% secties als dit de tekst verduidelijkt. Zaken die aan bod kunnen komen in de
% inleiding:


%   - context, achtergrond
%   - afbakenen van het onderwerp
%   - verantwoording van het onderwerp, methodologie
%   - probleemstelling
%   - onderzoeksdoelstelling
%   - onderzoeksvraag
%   - ...


\section{Problem Statement}
\label{sec:problem-statement}

% Uit je probleemstelling moet duidelijk zijn dat je onderzoek een meerwaarde
% heeft voor een concrete doelgroep. De doelgroep moet goed gedefinieerd en
% afgelijnd zijn. Doelgroepen als ``bedrijven,'' ``KMO's,'' systeembeheerders,
% enz.~zijn nog te vaag. Als je een lijstje kan maken van de
% personen/organisaties die een meerwaarde zullen vinden in deze bachelorproef
% (dit is eigenlijk je steekproefkader), dan is dat een indicatie dat de
% doelgroep goed gedefinieerd is. Dit kan een enkel bedrijf zijn of zelfs één
% persoon (je co-promotor/opdrachtgever).

The introduction of the \gls{microservice} and \gls{mfa} patterns gives rise to
technical and oraganizational challenges that were previously less commonly
encountered in the world of \gls{monolithic} development models. However, there
are significant benefits too if development teams are willing to adapt to these
methodologies. 

The research in this thesis can indicate whether development companies and teams
can benefit from a move towards one of these distributed architectures, or if
the benefits don't outweigh the difficulty of the changes.

Furthermore, while \textit{technology independence}, for example, can be
categorized as a benefit of said architecture patterns, this does not
automatically ensure that every technology is well suited. 

This thesis focusses on the specific characteristics of the Blazor \gls{wasm}
\gls{frontend} framework in relation to \glsplural{microfrontend}. Because of
the limited maturity of Blazor and the application of the \gls{mfa} pattern
within this tech stack, this thesis could be a valuable resource to companies
and development teams that have a focus on Blazor development or \.NET
technologies in general.

\section{Research question}
\label{sec:research-question}

% Wees zo concreet mogelijk bij het formuleren van je onderzoeksvraag. Een
% onderzoeksvraag is trouwens iets waar nog niemand op dit moment een antwoord
% heeft (voor zover je kan nagaan). Het opzoeken van bestaande informatie (bv.
% ``welke tools bestaan er voor deze toepassing?'') is dus geen onderzoeksvraag.
% Je kan de onderzoeksvraag verder specifiëren in deelvragen.

To be able to achieve the research objectives, the following research questions
were formulated:

\begin{itemize}
  \item[$RQ_1$] What is needed to be able to independently develop and deploy
  microfrontends using Blazor \gls{wasm}?
  \item[$RQ_2$] How to render Blazor-based \gls{microfrontend} with the proper
  isolation, performance and with \gls{pe} in mind?
  \item[$RQ_3$] What are the challenges that need to be overcome, and how would
  one do so? 
  \item[$RQ_4$] How can development teams benefit from the transformation of
  their Blazor monolith into a microfrontend solution?
\end{itemize}

\section{Research objective}
\label{sec:research-objective}

% Wat is het beoogde resultaat van je bachelorproef? Wat zijn de criteria voor
% succes? Beschrijf die zo concreet mogelijk. Gaat het bv. om een
% proof-of-concept, een prototype, een verslag met aanbevelingen, een
% vergelijkende studie, enz.

\begin{itemize}
  \item A comparative study providing an overview of the microfrontend
  architecture and the key differences with a monolithic architecture.
  \item A descriptive study outlining implementation patterns, challenges and
  best practices of the microfrontends pattern in a Blazor WebAssembly project.
  \item A descriptive study outlining the benefits and drawbacks of using the
  microfrontends architecture with the goal of enabling distributed
  development in a company context.
\end{itemize}

With the gained insight of the literature and the theoretical study, a
proof-of-concept solution around a realistic business case will be created to
investigate the feasibility and demonstrate the practical application of the
microfrontends architecture pattern in a Blazor \gls{wasm} project. 


\section{Structure of this bachelor thesis}
\label{sec:structure}

% Het is gebruikelijk aan het einde van de inleiding een overzicht te geven van
% de opbouw van de rest van de tekst. Deze sectie bevat al een aanzet die je kan
% aanvullen/aanpassen in functie van je eigen tekst.

De rest of this bachelor thesis is outlined as follows:

In Chapter~\ref{ch:state-of-the-art} an overview of the state of the art within
the research domain is provided. This overview is based on a literature study.

In Chapter~\ref{ch:methodology} the methodology is clarified and the relevant
research techniques are discussed to be able to formulate an answer to the
research questions.

% TODO: Own chapters...

Finally, in Chapter~\ref{ch:discussion}, the conclusions about the findings are
described and interpreted. Additionally, the significance of the results are
outlined, which could provide incentive for further research.