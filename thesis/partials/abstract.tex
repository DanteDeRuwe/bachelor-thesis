% ----------------------------------------
% Abstract
% ----------------------------------------

% Deze aspecten moeten zeker aan bod komen:
% - Context: waarom is dit werk belangrijk?
% - Nood: waarom moest dit onderzocht worden?
% - Taak: wat heb je precies gedaan?
% - Object: wat staat in dit document geschreven?
% - Resultaat: wat was het resultaat?
% - Conclusie: wat is/zijn de belangrijkste conclusie(s)?
% - Perspectief: blijven er nog vragen open die in de toekomst nog kunnen
%    onderzocht worden? Wat is een mogelijk vervolg voor jouw onderzoek?


% --- Dutch abstract ---

\IfLanguageName{english}{
\selectlanguage{dutch}
\chapter*{Samenvatting}
\selectlanguage{english}
}{}

% TODO

% --- Abstract ---

\chapter*{Abstract}
\label{ch:abstract}


With the \gls{wasm} \textit{(WASM)} standard a new set of web applications have
been made possible. Within the .NET ecosystem, the most popular solution for
generating \gls{wasm} applications is called Blazor. This framework allows
building interactive web UIs using C\# instead of JavaScript. One challenge in
this approach is that there is no direct way of enabling distributed
development. While component libraries can be created independently, knowledge
in the main application would be required for integration. Using a \gls{mfa}
this relationship could be inversed. 

This thesis investigates what is needed to empower the distributed development
of large-scale Blazor-based web applications. While the advantages of the
\gls{mfa} pattern are closely related with those of other distributed
architectures such as the \gls{ma}, a theoretical study uncovered that
\glsplural{microfrontend} have their own specific set of challenges that need to
be overcome. A proof-of-concept solution focussed around a realistic business
case was constructed, focussing on \gls{pe}. To achieve this, a universal
composition strategy was used. A reusable framework library was created that can
provide any Blazor application with components to achieve dynamic fragment
rendering and client-side routing, keeping also the debugging experience in
mind.

As the adoption of the Blazor framework and the \gls{mfa} pattern will mature,
this thesis could be a valuable starting point for further research, and a
valuable resource for .NET-focussed development teams
creating large-scale full-stack web applications. 