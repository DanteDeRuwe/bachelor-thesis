% ----------------------------------------
% Abstract
% ----------------------------------------

% Deze aspecten moeten zeker aan bod komen:
% - Context: waarom is dit werk belangrijk?
% - Nood: waarom moest dit onderzocht worden?
% - Taak: wat heb je precies gedaan?
% - Object: wat staat in dit document geschreven?
% - Resultaat: wat was het resultaat?
% - Conclusie: wat is/zijn de belangrijkste conclusie(s)?
% - Perspectief: blijven er nog vragen open die in de toekomst nog kunnen
%    onderzocht worden? Wat is een mogelijk vervolg voor jouw onderzoek?


% --- Dutch abstract ---

\IfLanguageName{english}{
\selectlanguage{dutch}
\chapter*{Samenvatting}

Door de introductie van de \gls{wasm} \textit{(WASM)} standaard is een hele
nieuwe waaier aan webapplicaties mogelijk. Binnen het .NET ecosysteem is Blazor
het meest gebruikte \textit{framework} om \gls{wasm} applicaties te genereren.
Dit framework maakt het mogelijk om interactieve \textit{user interfaces} te
bouwen, en hierbij gebruik te maken van C\# in de plaats van JavaScript. Een
uitdaging die bij deze manier van werken komt kijken, is dat er geen
rechtstreekse manier is om een gedistribueerde ontwikkeling mogelijk te maken.
Er kan gebruik gemaakt worden van onafhankelijke \textit{component libraries},
maar dan moet de uiteindelijke web applicatie kennis hebben van deze
\textit{libraries} bij de integratie. Een \textit{\gls{microfrontend}}
architectuur kan deze relatie potentieel omkeren.

Deze bachelorproef onderzoekt wat noodzakelijk is om Blazor applicaties
gedistribueerd en op grote schaal te ontwikkelen. Hoewel de voordelen van de
\textit{\gls{microfrontend}} architectuur grotendeels overlappen met die van
andere gedistribueerde architecturen zoals de \textit{\gls{microservice}}
architectuur, heeft een theoretisch onderzoek de uitdagingen die eigen zijn aan
de \textit{\gls{microfrontend}} architectuur kunnen blootleggen. Een
proof-of-concept applicatie werd daarna gecre\"eerd rond een specifieke
\textit{business case}, waarbij onder andere gefocust werd op zgn.
\textit{\gls{pe}}. Om dit te bereiken werd een universele compositiestrategie
gehanteerd. Een herbruikbare \textit{framework library} werd ook ontwikkeld, die
voor een Blazor applicatie enkele nuttige componenten kan bieden om dynamisch
zgn. \textit{fragments} weer te geven en \textit{client-side routing} mogelijk
te maken. Ook werd hierbij gedacht aan de mogelijkheid om te \textit{debuggen}.

Aangezien het gebruik van Blazor en de \textit{\gls{microfrontend}} architectuur
momenteel in de lift zit, kan deze bachelorproef een startpunt zijn voor verder
onderzoek, of een nuttige bron van informatie voor .NET ontwikkelbedrijven en
-teams die op grote schaal \textit{full-stack} webapplicaties ontwikkelen.


\selectlanguage{english}
}{}


% --- Abstract ---

\chapter*{Abstract}
\label{ch:abstract}


With the \gls{wasm} \textit{(WASM)} standard a new set of web applications have
been made possible. Within the .NET ecosystem, the most popular solution for
generating \gls{wasm} applications is called Blazor. This framework allows
building interactive web \glsplural{ui} using C\# instead of JavaScript. One
challenge in this approach is that there is no direct way of enabling
distributed development. While component libraries can be created independently,
knowledge in the main application would be required for integration. Using a
\gls{mfa} this relationship could be inversed. 

This thesis investigates what is needed to empower the distributed development
of large-scale Blazor-based web applications. While the advantages of the
\gls{mfa} pattern are closely related to those of other distributed
architectures such as the \gls{ma}, a theoretical study uncovered that
\glsplural{microfrontend} have their own specific set of challenges that need to
be overcome. A proof-of-concept solution focussed around a realistic business
case was constructed, focussing on \gls{pe}. To achieve this, a universal
composition strategy was used. A reusable framework library was created that can
provide any Blazor application with components to achieve dynamic fragment
rendering and client-side routing, keeping also the debugging experience in
mind.

As the adoption of the Blazor framework and the \gls{mfa} pattern will mature,
this thesis could be a valuable starting point for further research, and a
valuable resource for .NET-focussed development teams
creating large-scale full-stack web applications. 