% ----------------------------------------
% Proof of concept
% ----------------------------------------

\chapter{Proof of concept}
\label{ch:proof-of-concept}

In this chapter, a proof-of-concept solution around a specific business case is
created to demonstrate the use of th \gls{mfa} pattern.

As was made apparent in chapter \fullref{ch:state-of-the-art}, there are several
different ways of conceptualizing and creating \gls{microfrontend} solutions.
This thesis will focus on \textbf{universal composition}, as was described in
\fullref{sssec:universal-composition}, to be able to enable \textbf{\gls{pe}}.
The desired solution ideally uses an exclusive \textbf{.NET approach} (i.e.
using exclusively technologies within the .NET ecosystem), and has
\textbf{run-time integration} with dynamic (lazy) loading capabilities.


\section{Domain}
\subsection{Description}

This \gls{poc} is centered around an e-commerce domain. To make the
business case more specific and tangible, the case was formulated as
follows:

\begin{quote}
  The creation of a web application for an online store dedicated to selling
  tabletop games (board games, card games, ...).
\end{quote}

This encompasses the following functionalities:
\begin{itemize}
  \item Browse games
  \item View game details
  \item Order games
  % TODO?
\end{itemize}

This domain was chosen because the technical needs of e-commerce solutions tend
to overlap with the set-out desired characteristics of the proof-of-concept
application. The universal rendering will make sure the application has a faster
initial load and better \gls{seo} than a purely client-side rendered
application, while maintaining a highly dynamic nature.

\subsection{Decomposition into microfrontends}

According to \textcite{Rappl_2021}, the domain decomposition strategy is a
crucial foundation for the architecture of the \gls{microfrontend} solution. In
a real company context, both technical and organizational factors need to be
considered to be able to split up the domain. For this \gls{poc}, the following
modules were chosen:

\begin{itemize}
  \item The \textit{Discover} \gls{microfrontend}, containing the product
  overview and product details pages.
  \item The \textit{Order} \gls{microfrontend}, containing the order product
  button and the order page.
\end{itemize}

\section{Architecture}

Using the domain decomposition results, the overall architecture of the solution
can be determined. An overview is given in Figure~\ref{fig:poc-architecture}. It
mainly consists of 3 parts:
\begin{itemize}
  \item The \gls{appshell} where these \glsplural{microfrontend}
  will get integrated into. This will be discussed further in
  \fullref{sec:poc-development}.
  \item The \textit{Discover} \gls{microfrontend}
  \item The \textit{Order} \gls{microfrontend}
\end{itemize}
 

\begin{figure}
  \centering
  \includegraphics[scale=.8]{example-image}  %TODO 
  \caption[Architecture overview for proof-of-concept solution]{The architecture
  overview for the proof-of-concept solution. In the \glsplural{microfrontend},
  only the \gls{frontend} has been implementated in the \gls{poc}, while the
  other elements were mocked using in-memory solutions for demonstration
  purposes.
  }
  \label{fig:poc-diagram}
\end{figure}


\subsection{Composition structure}

A visual overview for the component and page composition for the \gls{poc} application is shown
in Figure~\ref{fig:poc-components}.

\begin{keeptogether}
  A summary of the components each \gls{microfrontend} exposes is shown below:

  \begin{tabular}{ll}
    \textit{Discover}  \gls{microfrontend}:&
    \begin{tabular}{ll}
      Pages & \texttt{GameDetails}\\
      Fragments & \texttt{GameOverview}
    \end{tabular}
    \\
    \textit{Order}  \gls{microfrontend}:&
    \begin{tabular}{ll}
      Pages & \texttt{Cart}, \texttt{OrderConfirmation}\\
      Fragments & \texttt{AddToCartButton}
    \end{tabular}
  \end{tabular}
\end{keeptogether}


\begin{figure}
  \centering
  \includegraphics[scale=.8]{example-image}  %TODO 
  \caption[Architecture overview for proof-of-concept solution]{A visual
  overview for the component and page composition for the \gls{poc}
  application}}
  \label{fig:poc-components}
\end{figure}




\section{Development}
\label{sec:poc-development}

