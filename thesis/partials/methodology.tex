% ----------------------------------------
% Methodology
% ----------------------------------------

\chapter{Methodology}
\label{ch:methodology}

%Hoe ben je te werk gegaan? Verdeel je onderzoek in grote fasen, en %
%licht in elke fase toe welke stappen je gevolgd hebt. Verantwoord waarom je %
%op deze manier te werk gegaan bent. Je moet kunnen aantonen dat je de best %
%mogelijke manier toegepast hebt om een antwoord te vinden op de %
%onderzoeksvraag.

The conducted research within this thesis can be subdivided into 2 parts, as to
accomodate the research objective:

\begin{itemize}
  \item A theoretical study
  \item A proof of concept
\end{itemize}

\section{Theoretical study}
To be able to answer the research questions, a theoretical study was conducted
in 4 phases:\\

\begin{tabularx}{\textwidth}{lX}
    Phase 1: & Investigating what distributed development entails, and what
    benefits and challenges it brings within a company context.\blankline\\
    Phase 2: & Providing an overview of how the \gls{mfa} pattern came to
    fruition historically, and laying out the key differences with other more
    \gls{monolithic} architectures.\blankline\\
    Phase 3: & Exploring characteristics, implementation patterns, benefits, challenges and
    best practices of the \gls{mfa} pattern.\blankline\\
    Phase 4: & Investigating the usage of the \gls{mfa} pattern in a Blazor \gls{wasm}
    project, and also looking at the possible challenges and their probable
    solutions.\blankline
\end{tabularx}

The existing literature was scoured in the form of academic papers, conferences,
books, blogposts and talks from renowned technology evangelists, etc... 

The results of this theoretical study were incorporated into
Chapter~\fullref{ch:state-of-the-art}. 


%%%

\section{Proof of concept}

To be able to answer the research objectives, a proof-of-concept application was
created using various online and offline resources to aid in the
conceptualization and development process. These resources included academic
papers, books and blogposts; as well as video presentations and conference
talks. Even open-source code found on GitHub\footnote{Code sources include the
ASP.NET Core repo found on \url{https://github.com/dotnet/aspnetcore} and the
Piral.Blazor repo found on \url{https://github.com/smapiot/Piral.Blazor}} was
used to find good solutions for the problems at hand.

Some of these concepts were tested seperately in a small ``dummy'' application
before being incorporated into the main solution, to ensure their robustness.

The development process and results of the proof-of-concept application can be
found in Chapter~\fullref{ch:proof-of-concept}.