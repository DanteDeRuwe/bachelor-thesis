%==============================================================================
% Bachelor Thesis Proposal by Dante De Ruwe
%==============================================================================
% Applied Information Technology 2020-2021 Faculty of IT and Digital
% Transformation @ Ghent University College Based on a LaTeX-template by Jens
% Buysse and Bert Van Vreckem 

\documentclass{proposal}

% ----------------------------------------
% Metadata
% ----------------------------------------

% --- Title and Author ---
\PaperTitle{Distributed Development of Blazor-based Web Applications using a
Universal Micro Frontend Architecture} \PaperType{Bachelor Thesis Proposal
2020-2021}

\Authors{Dante De Ruwe\textsuperscript{1}} \CoPromotor{Dr. Florian
Rappl\textsuperscript{2} (Solution Architect at Smapiot GmbH)}
\affiliation{\textbf{Contact:}
    \textsuperscript{1}
    \href{mailto:dantederuwe@gmail.com}{dantederuwe@gmail.com};
    \textsuperscript{2}
    \href{mailto:florian.rappl@smapiot.com}{florian.rappl@smapiot.com};}

% --- Abstract --- TODO Hier schrijf je de samenvatting van je voorstel, als een
% doorlopende tekst van één paragraaf. Wat hier zeker in moet vermeld worden: 
% - Context (Waarom is dit werk belangrijk?)
% - Nood (Waarom moet dit onderzocht worden?)
% - Taak (Wat ga je (ongeveer) doen?);
% - Object (Wat staat in dit document geschreven?)
% - Resultaat (Wat verwacht je van je onderzoek?)
% - Conclusie (Wat verwacht je van van de conclusies?)
% - Perspectief (Wat zegt de toekomst voor dit werk?).

\Abstract{With the WebAssembly \textit{(WASM)} standard a new set of web
    applications have been made possible. The technology stack selection no
    longer needs to be given as only Javascript to the browser. In the .NET
    ecosystem, the most popular solution for generating WASM applications is
    called Blazor. This framework allows writing full-stack managed code, which
    can then be used to make interactive clients, server-side rendering, and
    native mobile apps work. One challenge in this approach is that there is no
    direct way of enabling distributed development. While component libraries
    can be created independently, knowledge in the main application would be
    required for integration. Using a micro frontend architecture this
    relationship could be reversed. This thesis investigates what is needed to
    empower distributed development of large-scale Blazor-based web
    applications, tackling the individual challenges such as debugging or
    assembly sharing on its way.}

% --- Domain and Key Words ---
\newcommand{\keywordname}{Key Words} % Defines the keywords heading name
\Keywords{Web Development. Application Architecture -- Front-End -- Blazor --
    WebAssembly -- Micro Frontends} 

% ----------------------------------------
% Document
% ----------------------------------------

\begin{document}

\flushbottom            % Makes all text pages the same height
\maketitle              % Print the title and abstract box
\tableofcontents        % Print the contents section
\thispagestyle{empty}   % Removes page numbering from the first page


% --- Contents ---
% ----------------------------------------
% Inleiding 
% ----------------------------------------

\section{Introduction}
\label{sec:introduction}

% Hier introduceer je werk. Je hoeft hier nog niet te technisch te gaan. Je
% beschrijft zeker:
%   - de probleemstelling en context
%   - de motivatie en relevantie voor het onderzoek
%   - de doelstelling en onderzoeksvraag/-vragen

The development of applications for the web has seen some dramatic shifts over
the years. Apart from new technologies, protocols and standards, the way web
applications are structured has undergone some evolutions as well. ``Software
architecture'' not only outlines the pure structure of the application, but also
defines the responsibility of all the pieces of the application, and how these
pieces ought to interact with each other \autocite{Fedorov_etal_1998}.

In the ``early days'' of web development, nearly all applications had a
\textit{monolithic} architecture: a single-tier architecture, where the user
interface (UI), business logic and data storage are all managed in a single
all-in-one solution, managed by a big team. Now, developer teams have mostly
adopted split-stack development, where the UI is handled in the so-called
``front end'' and the business and data logic are dealt with by a ``back end''
system. This reduced coupling enabled specialized teams to develop each aspect
individually, independently and therefore simultaneously
\autocite{Dunkley_2016}. For the same reasons, \textit{microservices} started
making an appearance when developers realized that having a single backend
service could also be considered a monolithic approach
\autocite{Fowler_Microservices_2014}.

In 2016 the ThoughtWorks Technology Radar \autocite{ThoughtWorks_2020} coined
the term ``\textit{Micro Frontends}'' to describe the split of the front-end
monolith into independently deployable and maintainable pieces. This is
especially beneficial for large-scale projects. However, enabling distributed
development of these micro frontends is not a trivial undertaking, and naturally
has its challenges.

This thesis aims to be an application of this micro frontends pattern, to enable
distributed development of large scale web applications. More specifically, this
thesis will focus on the Blazor\hreffootnote{https://blazor.net} web framework, tackling the challenges and the
limitations that go along with this on its way.

\begin{minipage}{\linewidth} %to keep content together
     The research questions that will be addressed:
    \begin{itemize}
        \item[$RQ_1$] What is needed to be able to independently develop and deploy micro
        frontends using Blazor WebAssembly?
        \item[$RQ_2$] How to render Blazor-based micro frontends with the proper isolation,
        performance and with progressive enhancement in mind?
        \item[$RQ_3$] How can developer teams benefit from the transformation of their
        Blazor monolith into a micro frontend solution?
        \item[$RQ_4$] Wat are the challenges that need to be overcome, and how would one do
        so? 
    \end{itemize}
\end{minipage}

\blankline
Additionally, an objective of this thesis is to generate an open-source
proof-of-concept Blazor application that implements the micro frontend
architecture pattern.
% ---------------------------------------- 
% State-of-the-art 
% ----------------------------------------

\section{State-of-the-art}
\label{sec:state-of-the-art}

% Hier beschrijf je de \emph{state-of-the-art} rondom je gekozen
% onderzoeksdomein. Dit kan bijvoorbeeld een literatuurstudie zijn. Je mag de
% titel van deze sectie ook aanpassen (literatuurstudie, stand van zaken, enz.).
% Zijn er al gelijkaardige onderzoeken gevoerd? Wat concluderen ze? Wat is het
% verschil met jouw onderzoek? Wat is de relevantie met jouw onderzoek? Verwijs
% bij elke introductie van een term of bewering over het domein naar de
% vakliteratuur, bijvoorbeeld~\autocite{Doll1954}! Denk zeker goed na welke
% werken je refereert en waarom.
%
% Voor literatuurverwijzingen zijn er twee belangrijke commando's:
% \autocite{KEY} => (Auteur, jaartal) Gebruik dit als de naam van de auteur geen
% onderdeel is van de zin. \textcite{KEY} => Auteur (jaartal)  Gebruik dit als
% de auteursnaam wel een functie heeft in de zin (bv. ``Uit onderzoek door Doll
% & Hill (1954) bleek
%   ...'')
%
% Je mag gerust gebruik maken van subsecties in dit onderdeel.

\subsection{Distributed development and team structure}
As software projects grow larger and larger in size, naturally more developers
are needed to maintain them. But the efficiency of teams does not scale
linearly. This is a wisdom that dates back to the very early days of software
development, as proven by the still applicable quote by \textcite{Brooks_1975}:
``adding manpower to a late software project makes it later''.

A solution to this is obviously to subdivide the project into chunks, managed by
specialized teams. A company could even let these teams operate from a different
geographical location. This has some major benefits, however, significantly
increases the need for extensive coordination and communication between the
teams, as well as a good company structure and strategy. This all starts with
the choice of a good common architecture that is scalable and supports
distributed development \autocite{Yuhong_2008}. 

\subsection{Why Micro Frontends?}

As outlined in the introduction to this proposal, there is an already widespread
practice of splitting up projects ``horizontally'' per layer or technology. It
makes sure that experts can co-operate together as one team and ensure a high
technical standard within the boundaries of their respective areas of expertise.

But what if a company wants to put its focus on user experience, innovation and
features, instead of purely on creating the most technically perfect solution?
This is where multidisciplinary or cross-functional ``feature teams'' can come in.
These are grouped around a specific business case or customer need. This
``vertical slicing'' has major benefits: it enables teams to be completely
independent and have end-to-end responsibility for the features they have to
develop. This aids in accelerating development speed, cutting down on inter-team
communication and enabling developers to feel a greater sense of involvement in
the project or product \autocite{LarmanVodde_2008}.

The ``micro frontends'' architecture gives developers the necessary tools to be
able to organize themselves into autonomous ``feature teams'', where each team
has a company mission they specialize in.
\autocite{Geers_2020}

\subsection{Composing and integrating Micro Frontends}
Implementing the micro frontend architecture can be done in various ways. This
proposal will not mention all of them, nor will it provide a detailed look at
any of these techniques. However, below some common methodologies are outlined.
\autocite{Geers_2020} \autocite{Peltonen_etal_2020} \autocite{Pavlenko_etal_2020}

\begin{itemize}
    \item Page transitions via hyperlinks
    \item Composition via iframe
    \item Composition via AJAX and Web Components
    \item Server-side rendering
    \item Client-side rendering
    \item Universal rendering
\end{itemize}

This thesis will focus on universal (also known as
\textit{isomorphic}) rendering. This is a rendering technique that aims to
combine client-side and server-side rendering practices. The main concept is
pre-rendering the application on the server, resulting in a relatively fast
initial load, and then \textit{(re)hydrating} the pre-rendered application on the
client side to make it fully client-side interactive. This enables progressive
enhancement. \autocite{MillerOsmani_2019}

\subsubsection{Blazor WebAssembly}
...



% ---------------------------------------- 
% Methodology 
% ----------------------------------------

\section{Methodology}
\label{sec:methodology}

% Hier beschrijf je hoe je van plan bent het onderzoek te voeren. Welke
% onderzoekstechniek ga je toepassen om elk van je onderzoeksvragen te
% beantwoorden? Gebruik je hiervoor experimenten, vragenlijsten, simulaties? Je
% beschrijft ook al welke tools je denkt hiervoor te gebruiken of te
% ontwikkelen.

\subsection{Theoretical study}
To be able to answer the research questions a theoretical study will be
conducted in 3 phases:
\begin{itemize}
    \item Phase 1: Providing an overview of the micro frontend architecture and
    the key differences with a monolithic architecture.
    \item Phase 2: Exploring implementation patterns and best practices of the
    micro frontends pattern  in a Blazor WebAssembly project.
    \item Phase 3: Investigating the benefits, challenges and drawbacks of the
    micro service architecture in a company context.
\end{itemize}

\subsection{Proof-of-concept}
With the gained insight of the literature and the theoretical study, a
proof-of-concept solution around a realistic business case will be created to
demonstrate the practical application of the micro frontends pattern in a Blazor
WebAssembly project.

% ---------------------------------------- 
% Expected results 
% ----------------------------------------

\section{Expected Results}
\label{sec:expected-results}

% Hier beschrijf je welke resultaten je verwacht. Als je metingen en simulaties
% uitvoert, kan je hier al mock-ups maken van de grafieken samen met de
% verwachte conclusies. Benoem zeker al je assen en de stukken van de grafiek
% die je gaat gebruiken. Dit zorgt ervoor dat je concreet weet hoe je je data
% gaat moeten structureren.

% ---------------------------------------- 
% Expected conclusions 
% ----------------------------------------

\section{Expected conclusions}
\label{sec:expected-conclusions}

% Hier beschrijf je wat je verwacht uit je onderzoek, met de motivatie waarom.
% Het is niet erg indien uit je onderzoek andere resultaten en conclusies
% vloeien dan dat je hier beschrijft: het is dan juist interessant om te
% onderzoeken waarom jouw hypothesen niet overeenkomen met de resultaten.



% --- References ---
\phantomsection
\printbibliography[heading=bibintoc]

\end{document}