% ----------------------------------------
% Inleiding 
% ----------------------------------------

\section{Introduction}
\label{sec:introduction}

The development of applications for the web has seen some dramatic shifts over
the years. Apart from new technologies, protocols and standards, the way web
applications are structured has undergone some evolutions as well. ``Software
architecture'' not only outlines the pure structure of the application, but also
defines the responsibility of all the pieces of the application, and how these
pieces ought to interact with each other \autocite{Fedorov_etal_1998}.

In the ``early days'' of web development, nearly all applications had a
\textit{monolithic} architecture: a single-tier architecture, where the user
interface (UI), business logic and data storage are all managed in a single
all-in-one solution. Now, developer teams have mostly adopted split-stack
development, where the UI is handled in the so-called ``frontend'' and the
business and data logic are dealt with by a ``backend'' system. This reduced
coupling enabled specialized teams to develop each aspect individually,
independently and therefore simultaneously \autocite{Dunkley_2016}. For the same
reasons, \textit{microservices} started making an appearance when developers
realized that having a single backend service could also be considered a
monolithic approach \autocite{Fowler_Microservices_2014}.

In 2016 the ThoughtWorks Technology Radar \autocite{ThoughtWorks_2020} coined
the term ``\textit{Micro Frontends}'' to describe the split of the frontend
monolith into independently deployable and maintainable pieces. This is
especially beneficial for large-scale projects, where the division of developers
in autonomous teams is quite common. However, enabling distributed development
of these microfrontends is not a trivial undertaking, and naturally has its
challenges.

This thesis aims to be an application of the microfrontends architecture
pattern, to enable distributed development of large-scale web applications. More
specifically, this thesis will focus on the
Blazor\hreffootnote{https://blazor.net} web framework that uses
WebAssembly\hreffootnote{https://webassembly.org} to execute
C\#\hreffootnote{http://csharp.net} code in the browser.   

The research questions that will be addressed:
\begin{itemize}
    \item[$RQ_1$] What is needed to be able to independently develop and
    deploy microfrontends using Blazor WASM?
    \item[$RQ_2$] How to render Blazor-based microfrontends with the proper
    isolation, performance and with progressive
    enhancement\hreffootnote{https://wikipedia.org/wiki/Progressive\_enhancement}
    in mind?
    \item[$RQ_3$] What are the challenges that need to be overcome, and how
    would one do so? 
    \item[$RQ_4$] How can development teams benefit from the transformation
    of their Blazor monolith into a microfrontend solution?
\end{itemize}

Additionally, an objective of this thesis is to generate an open-source
proof-of-concept Blazor web application that implements the microfrontend
architecture pattern.



% ---------------------------------------- 
% State-of-the-art 
% ----------------------------------------

\section{State-of-the-art}
\label{sec:state-of-the-art}

\subsection{Distributed Development}
As software projects grow larger and larger in size, naturally more developers
are needed to maintain them. But the efficiency of teams does not scale
linearly. This is a wisdom that dates back to the very early days of software
development, as proven by the still applicable quote by \textcite{Brooks_1975}:
``adding manpower to a late software project makes it later''.\\
A solution to this is obviously to subdivide the project into chunks, managed by
specialized teams. A company could even let these teams operate from different
geographical locations. This is called ``distributed development''. While there
are major benefits; this significantly increases the need for extensive
coordination and communication between the teams, as well as a good company
structure and strategy. Making this possible all starts with the choice of a
good common architecture that is scalable and supports distributed development
\autocite{Yuhong_2008}. 

\subsection{Why Microfrontends?}
As outlined in the introduction to this proposal, there is an already widespread
practice of splitting up projects ``horizontally'' per layer or technology. It
makes sure that experts can co-operate together as one team and ensure a high
technical standard within the boundaries of their respective areas of expertise.

But what if a company wants to put its focus on user experience, innovation and
features, instead of purely on creating the most technically perfect solution?
This is where multidisciplinary or cross-functional ``feature teams'' can come
in. These are grouped around a specific business case or customer need. This
``vertical slicing'' has major benefits: it enables teams to be completely
independent and have end-to-end responsibility for the features they develop.
This aids in accelerating development speed, cutting down on inter-team
communication and enabling developers to feel a greater sense of involvement in
the project or product \autocite{LarmanVodde_2008}.

The ``microfrontends'' architecture gives developers the necessary tools to be
able to organize themselves into autonomous ``feature teams'', where each team
has a company mission they specialize in. \autocite{Geers_2020}

\subsection{Composing and Integrating Microfrontends}
Implementing the microfrontend architecture pattern can be done in various
ways. This proposal will not mention all of them, nor will it provide a detailed
look at any of these techniques. However, below some common methodologies are
outlined. \autocite{Geers_2020} \autocite{Peltonen_etal_2020}
\autocite{Pavlenko_etal_2020}

\begin{itemize}
    \item The web approach (hyperlinks, iframes...)
    \item Server-side rendering
    \item Client-side rendering
    \item Universal rendering
\end{itemize}

This thesis will focus on universal (also known as \textit{isomorphic})
rendering. This is a rendering technique that aims to combine client-side and
server-side rendering practices. The main concept is pre-rendering the
application on the server, resulting in a relatively fast initial load, and then
\textit{(re)hydrating} the pre-rendered application on the client side to make
it fully client-side interactive. This enables progressive enhancement.
\autocite{MillerOsmani_2019}

\subsection{Blazor WebAssembly}
According to \textcite{Rappl_MunichNETMeetup_2020}, an easy and effective way to
make distributed development possible in a Blazor WebAssembly project, is simply
to use separately distributed component libraries. In Blazor -- and in the .NET
ecosystem in general -- NuGet\hreffootnote{https://www.nuget.org} is the
standardized way of library distribution. One could create one overarching web
app which imports and incorporates these NuGet packages. This is often called
the \textit{app shell} \autocite{Geers_2020}. While this implementation pattern
makes development relatively straight-forward, it has its downsides. Because the
integration happens at build-time; any change requires full recompilation of the
entire application. Also, upon startup, all libraries have to be loaded, which
makes this approach sub-optimal regarding scalability. This integration method
re-introduces coupling, and is generally discouraged \autocite{Jackson_2019}.

Integrating Blazor components in a JavaScript based app shell is another option.
While this has been done succesfully\footnote{See an example here:
\hrefself{https://github.com/lauchacarro/MicroFrontend-Blazor-React}}, this is
not within the scope of this thesis, as the focus is on an almost exclusive
.NET-approach.

The desired solution, in contrast to the aforementioned approaches, uses
\textit{``C\#.NET all the way''} if possible, and has run-time integration with
dynamic (lazy) loading capabilities. In this, the challenges may come because of
the Blazor Intermediate Language (IL) trimming, which removes unused code and
libraries in the output assemblies \autocite{Latham_2020}. Luckily, .NET 5
introduces lazy loading capabilities for Blazor out of the box
\hreffootnote{https://docs.microsoft.com/en-us/aspnet/core/blazor/webassembly-lazy-load-assemblies}
\autocite{Kdouh_2020} \autocite{Rappl_dotNETConf_2020}, yet it is unclear if
this is the cure-all solution.

It bears repeating that progressive enhancement is desired. The combination of
these requirements will be achieved by using universal rendering, but with the
focus on the server-side.




% ---------------------------------------- 
% Methodology 
% ----------------------------------------

\section{Methodology}
\label{sec:methodology}

\subsection{Theoretical study}
To be able to answer the research questions, a theoretical study will be
conducted in 3 phases:
\begin{itemize}
    \item Phase 1: Providing an overview of the microfrontend architecture and
    the key differences with a monolithic architecture.
    \item Phase 2: Exploring implementation patterns, challenges and best
    practices of the microfrontends pattern in a Blazor WebAssembly project.
    \item Phase 3: Investigating the benefits and drawbacks of using the
    microfrontends architecture with the goal of enabling distributed
    development in a company context.
\end{itemize}

\subsection{Proof of concept}
With the gained insight of the literature and the theoretical study, a
proof-of-concept solution around a realistic business case will be created to
demonstrate the practical application of the microfrontends architecture pattern
in a Blazor WebAssembly project. Ideally, this proof of concept will be made
open-source.



% ---------------------------------------- 
% Expected results 
% ----------------------------------------

\section{Expected Results}
\label{sec:expected-results}

The expected results of the theoretical study are relatively straightforward. In
\textit{phase 1}, further insight will be gained into the microfrontend
architecture pattern. \textit{Phase 2} is predicted to provide valuable
knowledge for the eventual proof-of-concept solution. In particular, the
challenges that will undoubtedly come up in this phase, will be important
considerations. Some expected challenges include debugging, assembly sharing,
dynamic loading in combination with IL trimming ($\approx$ \textit{tree
shaking}), etc... With the results of research \textit{phase 3}, insight can be
gained into the cost-benefit balance of the microfrontends approach; and
indicate when and for whom it is appropriate. 

The proof of concept should be feasible to create, but well-thought-out
compromises will have to be made regarding the limitations of the current state
of the Blazor framework and WebAssembly standard.



% ---------------------------------------- 
% Expected conclusions 
% ----------------------------------------

\section{Expected Conclusions}
\label{sec:expected-conclusions}

The distributed development of Blazor-based web applications, using the
microfrontends pattern, with an almost exclusive .NET-approach should be
possible. Because of the complexity of integration, this is only beneficial to
large teams, or teams that expect scaling. Judging by the limited maturity of
Blazor and the application of the microfrontends pattern within this tech
stack, this thesis could provide one of the first realistic business cases
within this specific area.