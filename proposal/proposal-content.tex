% ----------------------------------------
% Inleiding 
% ----------------------------------------

\section{Introduction}
\label{sec:introduction}

% Hier introduceer je werk. Je hoeft hier nog niet te technisch te gaan.
% Je beschrijft zeker:
%   - de probleemstelling en context
%   - de motivatie en relevantie voor het onderzoek
%   - de doelstelling en onderzoeksvraag/-vragen

The development of applications for the web has seen some dramatic shifts over
the years. Apart from new technologies, protocols and standards, the way web
applications are structured have undergone some evolutions as well. ``Software
architecture'' not only outlines the pure structure of the application, but also
defines the responsibility of all the pieces of the application, and how these
pieces ought to interact with each other \autocite{Fedorov_etal_1998}.

In the ``early days'' of web development, nearly all applications had a
\textit{monolithic} architecture: a single-tier architecture, where the user
interface (UI), business logic and data storage are all managed in a single
all-in-one solution, managed by a big team. Now, developer teams have mostly
adopted split-stack development, where the UI is handled in the so-called
``front end'' and the business and data logic are dealt with by a ``back end''
system. This reduced coupling enabled specialized teams to develop
each aspect individually, independently and therefore simultaneously
\autocite{Dunkley_2016}. For the same reasons, \textit{microservices} started
making an appearance when developers realized that having a single backend
service could also be considered a monolithic approach
\autocite{Fowler_Microservices_2014}.

In 2016 the ThoughtWorks Technology Radar \autocite{ThoughtWorks_2020} coined
the term ``\textit{Micro Frontends}'' to describe the split of the front-end
monolith into independently deployable and maintainable pieces. This is
especially beneficial for large-scale projects. However, enabling distributed
development of these micro frontends is not a trivial undertaking, and naturally
has its challenges.

This thesis aims to be an application of this micro frontends pattern, to enable
distributed development of large scale web applications. More specifically, this
thesis will focus on the Blazor web framework, tackling the challenges and the
limitations that go along with this on its way.\\\\\noindent
The research questions that will be adressed:
\begin{itemize}
    \item What is needed to be able to independently develop and deploy micro
    frontends using Blazor WebAssembly?
    \item Wat are the challenges that need to be overcome, and how would one do
    so? 
    \item How to render Blazor-based microfrontends with the proper isolation,
    performance and with progressive enhancement in mind?
    \item How can developer teams benefit from the transformation of their
    Blazor monolith into a micro frontend solution?
    \item What are the disadvantages of this approach?
    \item[]
\end{itemize}

\noindent Aditionally, an objective of this thesis is to generate an open-source
proof-of-concept Blazor application that implements the micro frontend
architecture pattern.
% ---------------------------------------- 
% State-of-the-art 
% ----------------------------------------

\section{State-of-the-art}
\label{sec:state-of-the-art}

% Hier beschrijf je de \emph{state-of-the-art} rondom je gekozen onderzoeksdomein. 
% Dit kan bijvoorbeeld een literatuurstudie zijn. Je mag de titel van deze sectie ook aanpassen 
% (literatuurstudie, stand van zaken, enz.). Zijn er al gelijkaardige onderzoeken gevoerd? Wat concluderen ze? 
% Wat is het verschil met jouw onderzoek? Wat is de relevantie met jouw onderzoek?
% Verwijs bij elke introductie van een term of bewering over het domein naar de vakliteratuur, 
% bijvoorbeeld~\autocite{Doll1954}! Denk zeker goed na welke werken je refereert en waarom.
%
% Voor literatuurverwijzingen zijn er twee belangrijke commando's:
% \autocite{KEY} => (Auteur, jaartal) Gebruik dit als de naam van de auteur
%   geen onderdeel is van de zin.
% \textcite{KEY} => Auteur (jaartal)  Gebruik dit als de auteursnaam wel een
%   functie heeft in de zin (bv. ``Uit onderzoek door Doll & Hill (1954) bleek
%   ...'')
% 
% Je mag gerust gebruik maken van subsecties in dit onderdeel.

% ---------------------------------------- 
% Methodology 
% ----------------------------------------

\section{Methodology}
\label{sec:methodology}

% Hier beschrijf je hoe je van plan bent het onderzoek te voeren. 
% Welke onderzoekstechniek ga je toepassen om elk van je onderzoeksvragen te beantwoorden? 
% Gebruik je hiervoor experimenten, vragenlijsten, simulaties? Je beschrijft ook al welke tools je denkt 
% hiervoor te gebruiken of te ontwikkelen.

% ---------------------------------------- 
% Expected results 
% ----------------------------------------

\section{Expected Results}
\label{sec:expected-results}

% Hier beschrijf je welke resultaten je verwacht. 
% Als je metingen en simulaties uitvoert, kan je hier al mock-ups maken van de grafieken samen met de verwachte 
% conclusies. Benoem zeker al je assen en de stukken van de grafiek die je gaat gebruiken. Dit zorgt ervoor dat je 
% concreet weet hoe je je data gaat moeten structureren.

% ---------------------------------------- 
% Expected conclusions 
% ----------------------------------------

\section{Expected conclusions}
\label{sec:expected-conclusions}

% Hier beschrijf je wat je verwacht uit je onderzoek, met de motivatie waarom. 
% Het is niet erg indien uit je onderzoek andere resultaten en conclusies vloeien dan dat je hier beschrijft: 
% het is dan juist interessant om te onderzoeken waarom jouw hypothesen niet overeenkomen met de resultaten.

